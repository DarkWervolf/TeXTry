\documentclass[14pt,a4paper]{article}
\usepackage[utf8]{inputenc}
\usepackage[russianb]{babel}
\usepackage[left=1.5cm,right=1.5cm,top=2cm,bottom=2.5cm]{geometry}
\usepackage{setspace}
\usepackage{indentfirst}
\usepackage{amssymb}
\usepackage{amsmath}
\usepackage{array}
\usepackage[pdftex]{graphicx}
\usepackage{comment}


\graphicspath{{images/}}
\renewcommand{\baselinestretch}{1.3}


\begin{document}

\textbf{Задача 9}

Сколько существует различных пятеричных кодов длиной 6 символов, содержащих 3 двойки? Пятеричный код код обязательно начинается с двух единиц.

\textbf{Ответ}

16

\textbf{Решение}

Первым и вторым символом в пятеричном коде являются единицы. Три символа займут двойки, значит есть 4 варианта расстановки последней цифры: (на месте звёздочки может быть любая пятеричная цифра)

$$
\begin{array}{c}
	11*222 \\
	112*22 \\
	1122*2 \\
	11222* \\
\end{array}
$$

На место * можно поставить 4 символа: 0, 1, 3, 4, потому что нам нужно ровно 3 двойки, следовательно, мы не можем использовать её, потому что тогда их количество в коде может стать равно 4. Количество вариаций одного кода, допустим, первого, равно: $1 \cdot 1 \cdot 4 \cdot 1 \cdot 1 \cdot 1 = 4$. Всего таких кодов, чьи вариации не пересекаются, 4 (смотри таблицу), следовательно ответ: $4 \cdot 4 = 16$. 

\newpage

\end{document}